\section{Bottomonia production mechanism in heavy ion collisions}
\label{sec:Bottomonia_hi}

Quarkonia are predicted to be suppressed in heavy ion collisions
if QGP is formed since the force between the quarks will be colour screened
in QGP phase~\cite{Matsui:1986dk}.
 However, very soon it was realized that the picture was not that simple.
There are many factors which affect the production of quarkonia in A+A collisions. 
In fact, in proton-nucleus (p+A)
collisions also the quarkonium suppression was observed. That part of the
nucleus-nucleus suppression is due to 
cold-nuclear-matter (CNM) effects. Therefore it is necessary to disentangle hot 
and cold-medium effects. The CNM effects can arise at the initial
state and/or the final state. The initial state effect
arises due to modification of parton distribution functions (PDF) inside the nucleus
compared to the same inside the protons. The final state modification 
arises due to the  fact the produced quarkonia would interact with the medium
leading to the destabilisation of the bound state. Furthermore, the suppression of
quarkonia is thought to be of sequential in nature.  The sequential suppression
happens as a result of the differences of the  binding energy of different bound states. 
The strongly bound states, such as the $\upsa$ or the $\jpsi$,  melt at higher 
temperatures. On the other hand  more loosely bound staes \psiP, \chic, \chib, 
\upsb or \upsc  melt at much lower temperatures.  This behaviour helps  
 estimating the initial temperature reached in 
the collisions~\cite{Digal:2001ue}. However, the prediction of a sequential 
suppression pattern gets complicated due to feed-down 
decays of higher-mass resonances. The production process is further 
enriched, in the high energy scenario (like LHC), due to recombination process. At very 
high energies, abundant production of $Q$ and $\bar Q$ may lead to new quarkonia production 
in the medium. The recombination process is more justified for charmonia state and 
for the bottommonia states the contribution of this process is expected to be much
smaller since the bottom quark mass ($\sim$ 4.5 GeV) is three times more than the
charm quark and thus its thermalization at temperatures reached at LHC
($\sim$ 0.6 GeV) will be negligible. 



%In the following we discuss different contribution to the medium modification of quarkonia production. 


%\subsubsection{Spectral properties at high temperature}
%\label{sec:media_subsec31}
%There has been considerable interest in studying quarkonia in hot media as it was thought be a signature of the quark-hadron phase transition. 
%since publication of the famous Matsui and Satz 
 %paper~\cite{Matsui:1986dk}.



\subsection{Cold nuclear matter effects}
%{\color{red} This subsection will include the details of the models which use nuclear PDFs for quarkonia production
%like EPS19 or Color Glass Condensate modes. etc.~\cite{}. These effects are small for the Upsilon sector. }

The baseline for quarkonium production and suppression in heavy-ion collisions 
should be determined from studies of
cold-nuclear-matter (CNM) effects. The name cold matter 
arises because these effects are observed in hadron-nucleus interactions 
where dense matter effects are much more important compared to the hot matter.
The most important CNM affect is due to the modifications of the parton distribution
functions (PDF) in the nucleus compared to that in the nucleon.
It depends mainly on two parameters, 
the momentum fraction of the parton $(x)$ and the scale of the parton-parton 
interaction $(Q^2)$. The nuclear density modified parton distribution
function is known as nPDF and nPDF to PDF ratio,
$R_i(x,Q^2)=f_i^{p \epsilon A} (x, Q^2) /f_i^p  (x, Q^2)$
quantifies the modification due to nuclear effect. 
In the small $x$ regime $(x < 10^{-2})$, this ratio is less than unity
and is referred to as small-x shadowing. At intermediate  $x(\sim 0.1)$
the ratio shows a hump like structure, a phenomenon known as 
anti-shadowing. Around $x\approx 0.6$, one observes a dip which is known as EMC
effect. The dynamics of partons 
within the nuclei is affected by the parton saturation which is successfully
studied by color glass condensate. In the 
final state, the quarkonia bound state scatters and re-scatters inelastically
while passing through the nucleus. This leads to 
the breakup or absorption of the bound state which is estimated by the
inelastic cross-section of the quarkonia with the nucleon. 

The contributions to CNM effects look straightforward. However there are a several 
uncertainties associated with it. 
The measurement of  nuclear modifications of the quark densities are relatively 
well-understood in nuclear deep-inelastic scattering (nDIS). On the other hand 
 the modifications of the 
gluon density are not directly measured.  The scaling violations in nDIS is one of the  
ways to constrain the nuclear gluon density. Another constraint is provided by 
overall momentum conservation.  However, more direct probes of the gluon 
density are needed. The shadowing parametrizations we have in hand are derived from 
global fits to the nuclear parton densities. This 
gives wide variations in the nuclear gluon 
density, from almost no effect to very large shadowing at low-$x$, 
compensated by strong antishadowing around $x \sim 0.1$.  


The nuclear absorption survival probability depends on the absorption cross section 
of the quarkonium. There are more inherent uncertainties 
in absorption than in the shadowing parametrization.
Typically an absorption cross section is a fit to the $A$ dependence 
of quarkonium production in pA collision at a given energy. 
This is rather simplistic since it is not known whether the object traversing the
nucleus is a precursor 
color-octet state or a fully-formed color-singlet quarkonium state. The \jpsi absorption cross 
section at $y \sim 0$ is seen to decrease with energy, regardless of the chosen shadowing 
parametrization~\cite{Lourenco:2008sk}. 

The analyses of $\jpsi$ production in fixed-target interactions~\cite{Lourenco:2008sk} 
show that the effective absorption
cross section depends on the energy of the initial beam and the rapidity or
$x_F$ of the observed $\jpsi$.  One possible interpretation is that 
low-momentum color-singlet states can hadronize in the
target, resulting in larger effective absorption cross sections at lower
center-of-mass energies and backward $x_F$ (or center-of-mass rapidity).
At higher energies, the states traverse the target more rapidly so that
the $x_F$ values at which they can hadronize in the target move 
back from midrapidity toward more negative $x_F$.
Finally, at sufficiently high energies, the quarkonium states pass 
through the target before hadronizing, resulting in negligible absorption
effects.  Thus the {\it effective} absorption cross section decreases with 
increasing center-of-mass energy. This is  because of the fact that faster states are less likely 
to hadronize inside the target.

This is a very simplistic picture. In practice, cold-nuclear-matter effects 
(initial-state energy loss, shadowing, final-state breakup, {\it etc.}) 
depend differently on the quarkonium kinematic variables and the collision energy. 
It is clearly unsatisfactory to combine all these mechanisms into an {\it effective} 
absorption cross section, as employed in the Glauber formalism, 
that only evaluates final-state absorption. 
%Simply taking the $\sigma_{\rm abs}$ obtained from 
%the analysis of the pA data 
%and using it to define the Pb+Pb baseline is not sufficient. 
A better understanding of absorption requires more detailed knowledge of the 
production mechanisms which are not fully understood yet.


The nuclear modification factors and ratios of Upsilon states are measured by CMS
experiment covering wide kinematic regions.
In section~\ref{sectionpA}, Figure~\ref{fig:LHCpPb5} shows the $\Upsilon$(nS) nuclear
modification factor, $R_{pA}$,  as a function of transverse momentum $p_{T}$ 
and rapidity in p+Pb colisions at 5.02 TeV measured by CMS~\cite{CMS:2022wfi}.
It is observed that all three Upsilon states are suppressed in p+Pb collisions.
Moreover, it is noticed that the excited states are more suppressed as compared
to the ground state.
Since the shadowing effects are expected to be similar
on all the three states~\cite{Vogt:2015uba}
the measurements indicate final state effects on the Upsilon states which need
to be understood. 


\subsection{Quarkonium in hot medium}
\label{sec:media_sec3}

It has been argued that the color screening 
in a deconfined QCD medium will destroy $\QQbar$ bound states
at sufficiently high temperatures. If the binding radius of the heavy
quark bound state is  much greater than the screening radius, then one heavy 
quark gets screened from the other and the pair is broken~\cite{Abdulsalam:2012bw}.
As the temperature 
increases, the screening radius becomes smaller and smaller compared to the 
binding radius and the quarkonium states become more and more unstable. 
Although, this idea was proposed long ago, first principle QCD calculations, 
which go beyond qualitative arguments, have been performed quite recently. 
Such calculations include lattice QCD determinations of quarkonium 
correlators~\cite{Umeda:2002vr,Asakawa:2003re,Datta:2003ww,Jakovac:2006sf,Aarts:2007pk},
potential model calculations 
of the quarkonium spectral functions with potentials based on lattice 
QCD~\cite{Digal:2001ue,Wong:2004zr,Mocsy:2005qw,Mocsy:2004bv,Alberico:2006vw,Cabrera:2006wh,Mocsy:2007yj,Mocsy:2007jz},
also effective 
field theory approaches which justify potential models and reveal new medium 
effects~\cite{Laine:2007qy,Laine:2007gj,Laine:2008cf,Brambilla:2008cx}.  
Furthermore, better modeling of 
quarkonium production in the medium created by heavy-ion collisions has 
been achieved.   These new advancements make it possible to disentangle the cold
and hot-medium effects on the quarkonium states which is crucial for the 
interpretation of heavy-ion data. 



\begin{figure}[h]
   \begin{center}
      \includegraphics[width=0.6\textwidth]{Figures/Fig15_LattSingEnergy.png}
      \caption{(Color online) The singlet free energy versus quark separation 
        calculated in 2+1 flavour QCD on $16^3 \times 4$ lattices at different 
             temperatures~\cite{Petreczky:2009ip,Petreczky:2010yn}.  
      }
      \label{Fig:LatticeSingEner}
   \end{center}
\end{figure}
In Lattice Gauge Theory color screening is studied  by 
calculating the spatial correlation function of a static quark and
antiquark in a color-singlet state which propagates in Euclidean time 
from $\tau=0$ to $\tau=1/T$, where $T$ is the temperature.
The result of such calculations in the Lattice  with dynamical quarks have been
reported in Refs.~\cite{Petreczky:2009ip,Petreczky:2010yn,Kaczmarek:2002mc}.
The logarithm of the singlet
correlation function, also called the singlet free energy,
is shown for (2+1) flavour in Fig.~\ref{Fig:LatticeSingEner}. 
As expected, in the zero-temperature limit,
the singlet free energy coincides with the zero-temperature potential. 
Figure~\ref{Fig:LatticeSingEner} also illustrates that,
at sufficiently short distances, the singlet free energy is
temperature independent and is equal to the zero-temperature potential. 
The range of interaction decreases with increasing temperature.
For temperatures just above the transition temperature, $T_c$, the heavy-quark 
interaction range becomes comparable to the charmonium radius. Based on 
this general observation, one would expect that the charmonium
states, as well as the excited bottomonium states, do not remain bound at
temperatures just above the deconfinement transition, often referred to as 
dissociation or melting. 

%\subsubsection{Quarkonium spectral functions and quarkonium potential}
%\label{sec:media_subsec33}
In-medium quarkonium properties are encoded in the corresponding 
spectral functions, as is quarkonium dissociation
at high temperatures. Spectral functions are defined as
the imaginary part of the retarded correlation function of quarkonium
operators. Bound states appear as peaks in the spectral functions.
The peaks broaden and eventually disappear with increasing temperature which
signals the melting of the given quarkonium state.
The quarkonium spectral functions can be calculated in potential models 
using the singlet free energy from Fig.~\ref{Fig:LatticeSingEner} or with different 
lattice-based potentials obtained using the singlet free energy
as an input~\cite{Mocsy:2007yj,Mocsy:2007jz}. 
The results for quenched QCD calculations are shown in Fig.~\ref{Fig:QuarkoniaSpecFuncLattice}
for S-wave charmonium  and bottomonium 
spectral functions~\cite{Mocsy:2007yj}.
All charmonium states are dissolved in the deconfined phase while the bottomonium 1S
state may persist up to $T \sim 2T_c$. An upper bound on the dissociation temperature 
(the temperatures above which no
bound state peaks can be seen in the spectral function and bound state 
formation is suppressed) can be obtained from the analysis of the spectral 
functions. Conservative upper limits on the dissociation
temperatures for the different quarkonium states obtained from 
a full QCD calculation~\cite{Mocsy:2007jz} are given in Table~\ref{tab:LatticeDissTemp}.

\begin{figure}[]
   \begin{center}
      {\includegraphics[width=0.49\textwidth]{Figures/Fig16l_JPsi_SpecFuncLattQCD.png}}
      {\includegraphics[width=0.49\textwidth]{Figures/Fig16r_Y1S_SpecFuncLattQCD.png}}
      \caption{(Color online) The S-wave charmonium (left) and 
        bottomonium (right) spectral 
        functions calculated in potential 
        models. 
        Insets: correlators compared to lattice data.  
        The {\it dotted curves} are the
        free spectral functions. Figures are taken from Ref.~\cite{Mocsy:2007yj}.
      }
      \label{Fig:QuarkoniaSpecFuncLattice} 
   \end{center}
\end{figure}





\begin{table}[tb]
   \caption{Upper bounds on the dissociation 
             temperatures for different quarkonia states~\cite{Mocsy:2007jz}.
             }
   \label{tab:LatticeDissTemp}
   \setlength{\tabcolsep}{0.41pc}
   \begin{center}
      \begin{tabular}{ccccccc}
      \hline\hline
      %\rule[10pt]{-1mm}{0mm}
      State & $\chi_{cJ}(1P)$ & $\psi^{'}$ &J/$\psi$  &$\Upsilon(2S)$ & $\chi_{bJ}(1P)$ &$\Upsilon(1S)$ \\%[1.0mm]
      \hline 
      %\rule[10pt]{-1mm}{0mm}
      $T_{\rm diss}$ & $\le T_c$ & $\le T_c$ & $1.2T_c$ & $1.2T_c$ & $1.3T_c$ & $2T_c$\\ 
\hline\hline
\end{tabular}
\end{center}
\end{table}



%\subsubsection{Summary of hot medium effects}
%\label{sec:SummMedEff}
Potential model calculations based on lattice QCD and resummed 
perturbative QCD calculations conclude that all charmonium states and the
excited bottomonium states dissolve in the deconfined medium. This leads to 
the reduction of the quarkonium yields in heavy-ion collisions 
compared to the binary scaling of p+p collisions. Recombination and edge
effects, however, will produce a nonzero yield.


              
\subsection{Bottomonia suppression using Lattice QCD inspired potential model rates}

Bottomonia suppression has been studied using first-principle
calculation of the thermal widths of the states and considering 
momentum anisotropy of the plasma~\cite{Strickland:2011aa,Krouppa:2016jcl,Krouppa:2018lkt}.
In this work, the phase-space distribution of gluons in the local
rest frame is assumed to be 

\begin{equation} 
f({\bf x},{\bf p}) = f_{\rm iso}\left(\sqrt{{\bf p}^2+ \xi({\bf p}\cdot{\bf n})^2 }  / 
p_{\rm hard} \right) 
\label{distribution}
\end{equation} 
In the above equation $\xi$ is a measure of the degree of anisotropy of the plasma given as 
%\begin{equation}
$\xi = \frac{1}{2} \langle 
{\bf p}_\perp^2\rangle/\langle p_z^2\rangle -1$
%\end{equation} 
where $p_z$ and 
${\bf p}_\perp $ are the partonic longitudinal and transverse momenta in the local
rest frame, respectively. In equation \ref{distribution}, $p_{\rm hard}$ is the momentum  
scale of the particles and can be identified with the temperature
in an isotropic plasma. 

An approximate form of the real perturbative heavy quark potential as a function of
$\xi$ can be written as~\cite{Dumitru:2007hy} (for $N_c=3$ and $N_f=2$). 
\begin{eqnarray}
Re[V_{\rm pert}] &=& - \alpha \exp(-\mu r)/r \nonumber \\
\left(\frac{\mu}{m_D}\right)^{-4} &=&  
1 + \xi\left(1 + \frac{\sqrt{2}(1+\xi)^2\left(\cos(2\theta) - 1 \right)}{(2+ \xi)^{5/2}} \right) 
\label{eq:muparam}
\end{eqnarray}
where $\alpha = 4\alpha_s/3$, $m_D^2 = (1.4)^2 16 \pi \alpha_s  \, p_{\rm hard}^2/3$ is the isotropic
Debye mass and $\theta$ is the angle with respect to the beamline.  
The factor of $(1.4)^2$ accounts for higher-order corrections to the isotropic Debye 
mass \cite{Kaczmarek:2004gv}.

This perturbative potential, given in equation (\ref{eq:muparam}) is modified to include
the non-perturbative (long range) contributions. 
The modified real part of the potential is given as~\cite{Dumitru:2007hy} 


%
\begin{equation} 
\label{eq:repot}
Re[V] = -\frac{\alpha}{r} \left(1+\mu \, r\right) \exp\left( -\mu
\, r  \right) + \frac{2\sigma}{\mu}\left[1-\exp\left( -\mu
\, r  \right)\right] 
 - \sigma \,r\, \exp(-\mu\,r)- \frac{0.8 \, \sigma}{m_Q^2\, r} \, ,
\end{equation}
%
where the last term is a temperature- and spin-independent quark mass correction 
\cite{Bali:1997am} and $\sigma = 0.223$ GeV is the string tension.  Here  $\alpha$ 
is chosen to be  $0.385$ 
to match zero temperature
binding energy data for heavy quark states \cite{Dumitru:2007hy}.
The imaginary part of the potential is taken same as the perturbative heavy quark
potential up to linear order in $\xi$ 
%
\begin{equation} 
Im[V_{\rm pert}] = -\alpha p_{\rm hard} \biggl\{ \phi(\hat{r}) - \xi \left[\psi_1(\hat{r},
\theta)+\psi_2(\hat{r}, \theta)\right]\biggr\} ,
\label{eq:impot}
\end{equation}
%
where $\hat{r}=m_D r$ and $\phi$, $\psi_1$, and $\psi_2$ are defined in Ref.~\cite{Krouppa:2016jcl}. 

 The full model potential, given by $V = Re[V] + i Im[V]$, is used to 
solve the Schr\"odinger equation. 
Solution of the Schr\"odinger equation gives the real and imaginary parts of 
the binding energy of the states.  The imaginary part defines the instantaneous width of the state
$Im[E_{\rm bind}(p_{\rm hard},\xi)] \equiv -\Gamma_T(p_{\rm hard},\xi)/2$. 
The resulting width $\Gamma_T(\tau)$ implicitly depends on the initial temperature of the
system.

The following rate equation is used to account for in-medium bottomonia state decay,
%
\begin{equation} \label{eq:rate}
\frac{dn(\tau,{\bf x}_\perp,\varsigma)}{d\tau} = -\Gamma(\tau,{\bf x}_\perp,\varsigma)n(\tau,{\bf x}_\perp,\varsigma) ,
\end{equation}
%
where   $\tau = \sqrt{t^{2} - z^{2}}$ is the longitudinal proper time,  ${\bf x}_{\perp}$ is the the transverse coordinate and 
 $\varsigma = {\rm arctanh}(z/t)$ is the the spatial rapidity. The rate of decay is computed by~\cite{Strickland:2011aa}
%
\begin{eqnarray}
\Gamma(\tau, {\bf x}_{\perp}, \varsigma) = 
& 2Im[E_{\text{bind}}(\tau, {\bf x}_{\perp}, \varsigma)] & \ \ Re[E_{\text{bind}}(\tau, {\bf x}_{\perp}, \varsigma)] > 0 \\ 
& = \gamma_{\text{dis}} & \ \ Re[E_{\text{bind}}(\tau, {\bf x}_{\perp}, \varsigma)] \leq 0. 
%\end{cases}
\end{eqnarray}
%
The suppression factor $R_{AA}$ as a function of $p_T$ and centrality 
is obtained as follows
\begin{equation}
R_{AA}({\bf x}_\perp,p_T,\varsigma,b) =% 
\exp\!\left(-\bar{\gamma}({\bf x}_\perp,p_T,\varsigma,b) \right)
\end{equation}
where
\begin{equation}
 \bar{\gamma}({\bf x}_\perp,p_T,
\varsigma,b) \equiv \Theta(\tau_f-\tau_{\rm form}(p_T)) \int_{{\rm max}(\tau_{\rm form}(p_T),\tau_0)}^{\tau_f} 
d\tau\,\Gamma_T(\tau,{\bf x}_\perp,\varsigma,b) 
\end{equation}
  Here $\tau_{0}$ and $\tau_{f}$ are the initial and freeze out times of the plasma and 
$\tau_{\rm form}$ is the formation time of the bottomonium state. 
Finally, one averages
over ${\bf x}_\perp$ to obtain 
\begin{equation}
\langle R_{AA}(p_T,\varsigma,b) \rangle \equiv 
\frac{\int_{{\bf x}_\perp} \! d{\bf x}_\perp \, T_{AA}({\bf x}_\perp)\,%
  R_{AA}({\bf x}_\perp,p_T,\varsigma,b)} 
{\int_{{\bf x}_\perp} \! d{\bf x}_\perp \, T_{AA}({\bf x}_\perp)}
\end{equation} 

\begin{figure}[t]
\begin{center}
\includegraphics[width=0.5\textwidth]{Figures/Fig22_YnsRAA_NPart_StricklandModel.pdf}
\end{center}
\vspace{-7mm}
\caption{(Color online) Model calculations \cite{Krouppa:2018lkt} of the $R_{\rm AA}$
  of $\Upsilon$(1S) and $\Upsilon$(2S) as a function of $N_{\text{part}}$
  in Pb+Pb collisions at $\sqrt{s_{\rm NN}}$=5.02 TeV.   
  A comparison is made with the data from CMS experiment \cite{CMS:2018zza} at
  the LHC.}
\label{fig:raasep}
\end{figure}

Figure~\ref{fig:raasep} shows the model calculations~\cite{Krouppa:2018lkt}
of the $R_{\rm AA}$ of $\Upsilon$(1S) and $\Upsilon$(2S) as a function of
$N_{\text{part}}$  in Pb+Pb collisions at $\sqrt{s_{\rm NN}}$=5.02 TeV.   
  A comparison is made with the data from CMS experiment \cite{CMS:2018zza} at
  the LHC. It is shown that there is substantial 
suppression  of $\Upsilon(1S)$ and $\Upsilon$(2S) which are attributed to the in-medium decay.  
A  similar suppression pattern is observed for $\chi_{b1}$ which 
may be attributed to the finite formation time of the $\chi_{b1}$. 
  


\subsection{Gluon dissociation of quarkonia in dynamical medium}

{\color{black}
  The quarkonia can undergo both dissociation and recombination.
 The quarkonia population $N_{Q}$ evolution with proper time $\tau$ can be studied via
a kinetic equation~\cite{Thews:2000rj}
  \begin{equation}\label{eqkin}
    {dN_{Q} \over d\tau}  =  - \lambda_D  \rho_g N_{Q} + \lambda_F {N_{q \bar{q}}^{2} \over V(\tau)},
  \end{equation}
  where $V(\tau)$ is the volume of the deconfined spatial region.
The $\lambda_{D}$ is the dissociation rate obtained by the dissociation cross section averaged over 
the momentum distribution of gluons $\rho_g$ and $\lambda_{F}$ is the formation
rate obtained by the formation cross section 
averaged over the momentum distribution of heavy quark pair $q$ and $\bar{q}$. 
$N_{q \bar{q}}$ is the number of initial heavy quark pairs produced per event depending on the
centrality defined by the number of participants.
  The number of quarkonia at freeze-out time $\tau_f$ is given by the solution of Eq.~(\ref{eqkin}),
  \begin{equation}
    N_{Q}(p_T) = S(p_T) \,N_{Q}^{\rm PbPb}(p_T)+N_{Q}^F(p_T).
    \label{eqbeta}
  \end{equation}
  Here $N_{Q}^{\rm PbPb}(p_T)$ is the number of initially-produced quarkonia (including shadowing)
  as a function of $p_T$ and $S(p_T)$ is their survival probability from gluon collisions at freeze-out, 
  \begin{equation}
    S(p_T) = \exp \left( {-\int_{\tau_0}^{\tau_f}f(\tau) \lambda_{\rm D}(T,p_T)\,\rho_g(T)\,d\tau} \right).
  \end{equation}
  The temperature $T(\tau)$ and the QGP fraction $f(\tau)$ evolve from initial time $\tau_0$ 
  to freeze-out time $\tau_f$ due to expansion of the QGP. The initial temperature and the 
  evolution is dependent on collision centrality $N_{\rm part}$.
  $N_{Q}^F(p_T)$ is the number of regenerated quarkonia per event,
  \begin{equation}
    N_{Q}^F(p_T)=S(p_T)N_{q \bar{q}}^{2} \int_{\tau_0}^{\tau_f}{{\lambda_{\mathrm{F}}(T,p_T) \over V(\tau)\,S(\tau,p_T)} d\tau}.
  \end{equation}
  The nuclear modification factor ($R_{AA}$) then can simply be written as~\cite{Kumar:2014kfa, Kumar:2019xdj}
  \begin{equation}
    R_{AA}(p_T)=S(p_T) \, R(p_T) + \frac{N_{Q}^F(p_T)}{N_{Q}^{pp}(p_T)}.
    \label{raa}
  \end{equation}
  Here $R(p_T)$ is the shadowing factor.
%  $R_{AA}$ as a function of collision centrality, including regeneration, is
%  \begin{equation}
%    R_{AA}(N_{\rm part}) = \frac{\int_{p_{T\,\rm cut}} N_{Q}^{pp}(p_T)S(p_T)\, R(p_T) dp_T}{\int_{p_{T\,\rm cut}} N_{Q}^{pp}(p_T) dp_T} + 
%    \frac{\int_{p_{T\, \rm cut}} N_{Q}^F(p_T) dp_T}{\int_{p_{T\, \rm cut}} N_{Q}^{pp}(p_T) dp_T}
%    \label{raa2}
%  \end{equation}
%  Here $p_{T~{\rm cut}}$ defines the $p_T$ range for a given experimental acceptance.
%  $N_{Q}^{pp}(p_T)$ is the unmodified $p_T$ distribution of quarkonia obtained by NLO 
%  calculations and scaled to a particular centrality of the Pb+Pb collisions.

  The gluon dissociation rate can be obtained in the color dipole
approximation~\cite{Bhanot:1979vb} as a function of gluon energy, $q^0$ as
 \begin{equation}
    \sigma_{D}(q^{0}) = {8\pi \over 3} \, {16^2 \over 3^2} {a_0 \over m_q}  \frac{(q^0/\epsilon_0 - 1)^{3/2}} {(q^0/\epsilon_0)^5},
 \end{equation}
  where $\epsilon_0$ is the quarkonia binding energy and $m_q$ is the charm/bottom quark mass 
  and $a_0=1/\sqrt{m_q\epsilon_0}$.
  The value of $\epsilon_0$ is equal to $1.10$ GeV for $\Upsilon$(1S) \cite{Karsch:1987pv}. 
For the first excited state of bottomonia, $\Upsilon$(2S), we use dissociation
 cross section from Ref.~\cite{Arleo:2001mp}.

  \begin{figure}
    \begin{center}
    \includegraphics[width=0.50\textwidth]{Figures/Fig17_Y1S_SigmaDq0.pdf}
    \caption{(Color online) Gluon dissociation cross section of $\Upsilon$(1S) as a
      function of gluon energy ($q^{0}$) in $\Upsilon$(1S) rest frame.}
    \label{fig:SigmaDQ0}
    \end{center}
  \end{figure}

  
  Figure \ref{fig:SigmaDQ0} shows the gluon dissociation cross sections of
$\Upsilon$(1S) as a function of gluon energy. The dissociation cross section
is zero when the gluon energy is less than the binding energy of the quarkonia.
It increases with gluon energy and reaches a maximum at 1.5 GeV for 
$\Upsilon(1{\rm S})$. At higher gluon energies, the interaction
probability decreases. We calculate the dissociation rate as a function of quarkonium
momentum by integrating the dissociation cross section over thermal gluon momentum 
distribution $f_{g}(p_g)$.


 We can calculate the formation cross section from the dissociation cross section using
detailed balance~\cite{Thews:2000rj,Thews:2005vj},
  \begin{equation}
    \sigma_{F} = \frac{48}{36}\,\sigma_{D}(q^0)\frac{(s-M_{Q}^2)^{2}}{s(s-4m_q^{2})}.
  \end{equation}
  The formation rate of quarkonium with momentum {\bf p} can be obtained using
  thermal distribution functions of  $q/\bar{q}$.
  
  

%%%%%%%%%%%%%%%%%%%%%%%%%%%%%%%%%%%%%%%%%%%%%%%%%%%%%%%% 5.02 TeV %%%%%%%%%%%%%%%%%%%%%%%%%%%%%%%%%%%%%%%%%
\begin{figure}
\includegraphics[width=0.49\textwidth]{Figures/Fig18l_Y1S_CMS_RAAPt_Shade.pdf}
\includegraphics[width=0.49\textwidth]{Figures/Fig18r_Y2S_CMS_RAAPt_Shade.pdf}
\caption{(Color online) Calculated nuclear modification factor ($R_{AA}$) \cite{Kumar:2019xdj}
  of (a) $\Upsilon$(1S) and 
  (b) $\Upsilon$(2S) as a function of $p_{T}$ 
  compared with CMS measurements~\cite{CMS:2018zza}.
The global uncertainty in $R_{AA}$ is shown as a band around the line at 1.
}
\label{fig:UpsilonRaaPtCMS}
\end{figure}



\begin{figure}
\includegraphics[width=0.49\textwidth]{Figures/Fig19l_ALICE_Y1SRAAPt_Shade.pdf}
\includegraphics[width=0.49\textwidth]{Figures/Fig19r_ALICE_Y2SRAAPt_Shade.pdf}
\caption{(Color online) Calculated nuclear modification factor ($R_{AA}$) \cite{Kumar:2019xdj}
  of (a) $\Upsilon$(1S) and 
  (b) $\Upsilon$(2S) as a function of $p_{T}$ in the kinematic range of ALICE detector
at LHC ~\cite{ALICE:2020wwx}. The global uncertainty in $R_{AA}$ is shown as a band
around the line at 1.
} 
\label{fig:UpsilonRaaPtALICE}
\end{figure}

\begin{figure}
\includegraphics[width=0.49\textwidth]{Figures/Fig20l_CMS_Y1SRAANPart_Shade.pdf}
\includegraphics[width=0.49\textwidth]{Figures/Fig20r_CMS_Y2SRAANPart_Shade.pdf}
\caption{(Color online) Calculated nuclear modification factor ($R_{AA}$) \cite{Kumar:2019xdj} of 
  (a) $\Upsilon$(1S) and (b) $\Upsilon$(2S) as a function of centrality of the 
  collisions compared with the CMS measurements~\cite{CMS:2018zza}.%\cite{CMS:2017ucd}.
  The global uncertainty in $R_{AA}$ is shown as a band around the line at 1.
}
\label{fig:UpsilonRaaNPartCMS}
\end{figure}


Figure~\ref{fig:UpsilonRaaPtCMS}(a) and (b) show the calculations~\cite{Kumar:2019xdj}
of various contributions to
the nuclear modification factor, $R_{AA}$, for the $\Upsilon$(1S) and $\Upsilon$(2S)
respectively as a function of $p_T$ compared with the mid rapidity measurements from
CMS~\cite{CMS:2018zza}.  
The gluon dissociation mechanism combined with the pion dissociation and shadowing
corrections gives good description of data in $p_{T}$ range ($p_{T}\approx$ 0-15 GeV/c)
for both $\Upsilon$(1S) and $\Upsilon$(2S).
The contribution from the regenerated $\Upsilon$s is negligible even at LHC energies.
The calculations under-predict the suppression observed at the highest measured
$p_{T}$ for $\Upsilon$(1S) and $\Upsilon$(2S) which is similar for the case
of J/$\psi$.
%The feed-down corrections are applied in our calculations following the similar
%procedure as in Refs.~\cite{Abdulsalam:2012bw,Krouppa:2017jlg}. 
%%%%%%%%% insert the feed-down details here

The feeddown corrections in the states $\Upsilon$(1S) and $\Upsilon$(2S) 
from decays of higher b$\bar{\rm b}$ bound states are obtained as
  \begin{equation}
    R_{AA}^{\Upsilon(3S)} = R_{AA}^{\Upsilon(3S)} %\nonumber
  \end{equation}
  \begin{equation}
    R_{AA}^{\Upsilon(2S)} = f_1 R_{AA}^{\Upsilon(2S)} +  f_2 R_{AA}^{\Upsilon(3S)}  %\nonumber
  \end{equation}
   \begin{equation}
    R_{AA}^{\Upsilon(1S)} = g_1 R_{AA}^{\Upsilon(1S)} +  g_2 R_{AA}^{\chi_b(1P)} + g_3 R_{AA}^{\Upsilon(2S)} + g_4 R_{AA}^{\Upsilon(3S)} %\nonumber
  \end{equation}
The factors $f$'s and $g$'s are obtained from CDF measurement~\cite{Affolder:1999wm}.
The values of $g_1$, $g_2$, $g_3$ and $g_4$ are 0.509, 0.27, 0.107
and 0.113 respectively. Here $g_4$ is assumed to be the combined fraction of 
$\Upsilon$(3S) and $\chi_b$(2P).
The values of $f_1$ and $f_2$ are taken as 0.50~\cite{Strickland:2011aa}.


Figure~\ref{fig:UpsilonRaaPtALICE}(a) and (b) show the model 
prediction \cite{Kumar:2019xdj} of the nuclear modification factor, $R_{AA}$, for the $\Upsilon$(1S)
and $\Upsilon$(2S) respectively as a function of $p_T$ in the kinematic range
covered by ALICE detector. The ALICE data~\cite{ALICE:2020wwx} is well described by the model.

Figure~\ref{fig:UpsilonRaaNPartCMS}(a) depicts the calculated \cite{Kumar:2019xdj}
centrality dependence of the $\Upsilon$(1S) nuclear
modification factor, along with the midrapidity data from CMS~\cite{CMS:2018zza}.
The calculations combined with the pion dissociation and shadowing corrections 
gives very good description of the measured data. Figure~\ref{fig:UpsilonRaaNPartCMS}(b)
shows the same for the $\Upsilon$(2S) along with the midrapidity
CMS measurements. The suppression of the excited $\Upsilon$(2S) states 
is also well described by the model. As stated earlier, the effect of regeneration is
negligible for $\Upsilon$ states. 

To summarise, the gluon dissociation mechanism combined with the shadowing
corrections gives very good description of data in mid $p_{T}$ range ($p_{T}\approx$ 5-10 GeV/c)
for both $\Upsilon$(1S) and $\Upsilon$(2S).
The contribution from the regenerated $\Upsilon$s is negligible even at LHC energies.
The calculations under-predict the suppression observed at the highest measured
$p_{T}$ for $\Upsilon$(1S) and $\Upsilon$(2S) which is similar for the case
of J/$\psi$.


  The suppression of quarkonia by comoving pions can be calculated by folding the quarkonium-pion
dissociation cross section $\sigma_{\pi Q}$ over thermal pion distributions \cite{Vogt:1988fj}. 
It is expected  that at LHC energies, the comover cross section will be small~\cite{Lourenco:2008sk}.
{\color{black}
The pion-quarkonia cross section is calculated by convoluting the gluon-quarkonia cross section $\sigma_D$
over the gluon distribution inside the pion~\cite{Arleo:2001mp},
\begin{equation}
\sigma_{\pi Q} (p_{\pi}) = {p_+^2 \over 2(p_\pi^2 - m_\pi^2)} \int_0^1 \, dx \, G(x) \, \sigma_D(xp_+/\sqrt {2}),
\end{equation}
where $p_+ = (p_\pi + \sqrt{p_\pi^2-m_\pi^2})/\sqrt{2}$. The gluon distribution, $G(x)$, inside a pion is 
given by the GRV parameterization~\cite{Gluck:1991ey}. 
The dissociation rate $\lambda_{D_{\pi}}$  can be obtained using the 
thermal pion distribution.




\subsection{Transport approach for bottomonia in the medium}
 The studies in Refs.~\cite{Grandchamp:2005yw,Rapp:2017chc} use 
transport approach for the bottomomia production in the medium~\cite{Grandchamp:2005yw,Rapp:2017chc}.
The rate equation for bottomonium evolution in the medium's rest frame
can be written as,
\begin{equation}
\frac{\mathrm{d} N_Y(\tau)}{\mathrm{d}\tau} =
-\Gamma_Y(T)\left[N_Y(\tau)-N^{\rm eq}_Y(T)\right] \ ,
\end{equation}
Here $\Gamma_Y$, is the inelastic reaction rate  and $N^{\rm eq}_Y(T)$ is the thermal
equilibrium limit  for each state $Y=\Upsilon(1S), \Upsilon(2S), \chi_c$.
In the reaction rates  both gluo-dissociation and quasi-free mechanisms have
been incorporated.  An important ingredient in this calculation is the bottomonium
binding energies. The thermal-equilibrium limit is evaluated from the statistical
model with bottom quarks~\cite{Grandchamp:2002wp}. 
The initial conditions are obtained from the p+p collision data. With these inputs,
the study is carried out in a hydrodynamicaly 
expanding scenario.  




\begin{figure}[t]
\includegraphics[width=0.49\textwidth]{Figures/Fig21l_YnsRAA_NPart_RappModel.pdf}
\includegraphics[width=0.49\textwidth]{Figures/Fig21r_YnsRAA_Pt_RappModel.pdf}
\caption{(Color online) Centrality (left) and transverse-momentum (right) dependence of the $R_{\rm AA}$~\cite{Rapp:2017chc}
  for $\Upsilon(1S)$ and $\Upsilon(2S)$ in 5.02\,TeV Pb-Pb collisions at the LHC,
  compared to CMS data~\cite{Flores:2017qmcms}.
  The bands represent a 0-15\,\% shadowing~\cite{Eskola:2009uj} on open-bottom and bottomonia.}
\label{fig_cms}
\end{figure}

Figure \ref{fig_cms} shows the Centrality (left) and transverse-momentum (right)
dependence of the $R_{\rm AA}$ calculated by model in Ref~\cite{Rapp:2017chc}
for $\Upsilon(1S)$ and $\Upsilon(2S)$ in 5.02\,TeV Pb-Pb collisions at the LHC,
compared to CMS data~\cite{Flores:2017qmcms}.
The authors of this model found a reasonable agreement  with experimental data
for the centrality dependence of both $\Upsilon(1S)$ and $\Upsilon(2S)$ at both
collision energies. Interestingly, they could reproduce the strong suppression of
the $\Upsilon(2S)$ observed by STAR.  
The calculated $p_T$ spectra at 5.02\,TeV appear to capture the rather flat
shapes in the CMS data at high $p_T$. 





