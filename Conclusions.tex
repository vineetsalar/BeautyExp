\section{Summary and Conclusions}
\label{sec:conclusions}
In this writeup, we have reviewed the field of bottomonia production in
p+p, p+A and A+A collisions. With immense experimental and theoretical actvities
specially due to LHC measurements, many features of the bottomonia production
and their behaviour in medium are well undersood.

In section 2, we have reviewed the experimental status of the bottomonia production
in p+p collisions. The measurements at Tevatron, by CDF and D0 collaborations,
have been discussed. The measurements at LHC, by CMS and ATLAS, at ${\sqrt s}=$ 7 TeV and
13 TeV have been reviewed. There have been some measurements 
of $\Upsilon$ polarization by CDF collaboration.
The CMS and LHCb data, on $\Upsilon$ polarizability, confirms 
negligible polarization. 
The measurements of the cross sections and polarizations have shed light on the
$\Upsilon$(1S, 2S, 3S) production mechanisms in p+p collisions.
LHC data has substantially extended the reach of the kinematics to test the
Non-Relativistic QCD (NRQCD) and other models with 
higher-order corrections which becomes more
distinguishable at higher transverese momenta.


In section 3, we have discussed theoretical models of the bottomonia production
mechanism in p+p collisions. The bottomonia study in p+p involves heavy quark
pair production treatable by perturbative process and the formation of bottomonia 
which is a non-perturbative process.
For the later one has to take recourse to some effective models. We have discussed the 
color singlet model, the color evaporation model and the NRQCD factorisation approach.
In color singlet model, it is assumed that the $Q\bar Q$ pair that evolves into
the quarkonium is in a color-singlet state. 
On the other hand in the color evaporation model,
it is assumed that every produced $\QQbar$ pair can evolve into a quarkonium,
if it has an invariant mass that is less than the threshold for
producing a pair of open-flavor heavy mesons.
The probability factor of a pair evolving into quarkonium is obtained by fitting with
the experiments which is supposed to be independent of collision energy. 
The so-called Improved CEM reproduces the tranverse momentum dependence of the
quarkonium cross section at CDF and LHC energies.  
We also presented NRQCD model formalism in detail.
The NRQCD formalism, along with color singlet state, includes the colour octet state.
In this formalism the evolution probability of $Q\bar{Q}$
pair into a state of quarkonium is expressed as matrix elements of NRQCD operators
expanded in terms of heavy quark velocity $v$.
The work using NLO cross sections are discussed and LO calculations have been
reproduced for $\Upsilon$(ns)
production in $p+p$ collision at $\sqrt s = 7$ and 13 TeV.

In section 4, we have presented an experimental overview of the bottomonia
results in p+A and A+A collisions at RHIC and LHC. We have looked into the
$R_{AA}$ for $\Upsilon$(ns) as a functions of kinematic variables 
$p_T$, $y$ and $N_{Part}$ at different energies and by different experiments. 
We have also studied $v_2$ for these states with centrality and $p_T$.
LHC has provided high statistics measurements of $R_{AA}$ for
Pb+Pb collisions for all three Upsilon states over wide kinematical ranges.
All $\Upsilon$ states are found to be suppressed in the Pb+Pb collisions,
the heavier states are more suppressed relative to the ground state.
The suppression of $\Upsilon$ states strongly depends on system size but
has weak dependence on $\pT$ and rapidity. At high $p_T$, more precise
measurements are required to ascertain flatness in the suppression. 
Comparing the measurements at RHIC and at two energies of LHC, it can be
said that the suppression increases with energy albiet weakly.

All the three Upsilon states are suppressed in p+Pb collisions as well
and the excited states are more suppressed than the ground state indicating
final state effects.
We have obtained a new figure for the ratio $\Upsilon$(2S)/$\Upsilon$(1S)
as a function of event activity measured in p+Pb and Pb+Pb collisions at
$\sqrt{s_{\rm NN}}$=5.02 TeV compared with the 
p+p collisions at $\sqrt{s}$=7 TeV. This study shows 
that the ratio $\Upsilon$(2S)/$\Upsilon$(1S) decreases steadily
with with increasing $N_{\rm tracks}$ for
p+p and p+Pb systems and the peripheral Pb+Pb data also follow this trend.
Then there is a step and most central Pb+Pb
data show a flatness as a function of event activity contrary to p+p and
p+Pb collisions which fall steadily with increasing $N_{\rm tracks}$.
This behaviour can be used to distinguish the Pb+Pb collision
system with the smaller systems.
 
No significant $v_2$ is found for $\Upsilon$(1S) measured by CMS
experiment. This shows that the bottom quark is not thermalized in the medium.
The recombination yield of bottomonia calculated using our model
is very small. 


In section 5, we have discussed the mechanisms for the modification of 
bottomonia yields in heavy ion collisions.
Starting with the idea of colour screening we have discussed the more recent
ideas like modification of spectral functions of quarkonia states as
a function of temperature.
The cold nuclear matter has been reviewed in certain amount of detail.
The excited bottomonia states are more suppressed as compared to ground state,
an effect which can not be produced by shadowing effect. The final state
effects in p+A collisions require a better theoretical understanding. 
We have reviewed some if not all the theoretical models treating both
the quarkonia dissociation and recombination in dynamical medium.
The comparison of the theoretical results with the experiments show that
the bottomonia production can be understood in terms of colour screening
or gluon dissociation. There is no significant recombination needed,
a picture which is also consistent with the small values of $v_2$ measured
by the experiments.

 

