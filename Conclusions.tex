\section{Summary and Conclusions}
\label{sec:conclusions}
In this writeup we have reviewd experimental and theoretical developments of
bottomonia production in proton-proton, proton-nucleus and nucleus-nucleus collisions.

In section 2, we have reviewed the experimental status of the bottomonia production in p-p collisions. The 
measurements at Tevatron, by CDF and D0 collaborations, have been discussed. The measurements at 
LHC, by CMS and ATLAS, at ${\sqrt s}=$ 7 TeV have been reviewed. There have been some measurements 
of $\Upsilon$ polarization by CDF collaboration.
The CMS and LHCb data, on $\Upsilon$ polarizability, confirms 
negligible polarization. 

In section 3 we have reviewed the bottomonia production mechanism in p-p 
collisions. The bottomonia or in general quarkonia production in p-p or A-A collisions consists of two 
sub processes. First, the heavy quark pairs are produced through a perturbative process and then 
these heavy quarks form quarkonia through a non-perturbative process. We have discussed the basic 
aspects of the perturbative production of heavy quarks. Since the formation of bottomonia 
is a non-perturbative process one has to take recourse to some effective models. We have discussed the 
color singlet model, the color evaporation model and the NRQCD factorisation approach. In color singlet  
model  it is assumed that the $Q\bar Q$ pair that evolves into
the quarkonium is in a color-singlet state. It is further assumed  that  $Q\bar Q$ pair   has the same spin
and angular-momentum quantum numbers as the quarkonium. On the other hand in the color evaporation 
model   it
is assumed that every produced $\QQbar$ pair evolves into a quarkonium
if it has an invariant mass that is less than the threshold for
producing a pair of open-flavor heavy mesons. The CEM prediction is found to be in good agreement 
with the CDF data. We have studied NRQCD formalism in quite amount of detail. The NRQCD formalism, 
along with color singlet state, includes the colour octet state. In this formalism the evolution probability of $Q\bar{Q}$
pair into a state of quarkonium is expressed as matrix elements of NRQCD operators expanded
in terms of heavy quark velocity $v$. We have discussed the $\Upsilon$(ns) production $p+p$ collision 
at $\sqrt s = 13$ TeV and at $\sqrt s =7$ TeV.

In section 4 we have presented an experimental overview of the bottomonia results at RHIC and LHC. 
We have looked into the $R_{AA}$ for $\Upsilon$(ns) as a functions of $p_T$, $y$ and $N_{part}$. 
We have also studied $v_2$ for these states with centrality and $p_T$. 

In section 5 we have discussed the production of bottomonia in heavy ion collisions. We have started the first proposal of 
Matsui and Satz for the suppression of quarkonia in the medium. Then we have discussed the more recent ideas like 
sequential suppression. The cold nuclear matter has been reviewed in certain amount of detail. In section 5.3 
we have discussed the kinetic approach. We have presented the theoretical results of $R_{AA}$ and compared those with the 
experimentally available data from CMS and ALICE. In section 5.4 we have discussed the transport approach in the 
bottomonia production. One again the results have been compared with the CMS data. Section 5.5 deals with the 
suppression in the anisotropic medium. The results of $R_{AA}$ in the anisotropic medium have been presented 
and compared with the CMS data. 

