\section{Introduction}
\label{sec:Introduction}
It is expected that strongly-interacting matter shows qualitatively
new behavior at temperatures and/or densities which are
comparable to or larger than the typical hadronic scale.
It has been argued that under such extreme conditions
deconfinement of quarks and gluons should set in and the 
thermodynamics of strongly-interacting matter could then
be understood in terms of these elementary degrees of freedom.
This new form of matter is called
{\em quark-gluon plasma}~\cite{Shuryak:1980tp,Satz:2011wf}, or QGP.
The existence of such a transition has indeed been demonstrated 
from first principles using Monte Carlo simulations of lattice QCD.
The deconfinement transition and the properties of hot, strongly-interacting 
matter can be studied experimentally in heavy-ion collisions~\cite{Satz:2000bn}. 
A significant part of the extensive experimental heavy-ion
program is dedicated to measuring quarkonium yields since Matsui and Satz
suggested that quarkonium suppression could be a signature of 
deconfinement~\cite{Matsui:1986dk}.

In fact, the observation of anomalous suppression was considered to be
a key signature of deconfinement at SPS energies~\cite{Kluberg:2005yh}.
