\newpage
\section{Introduction}
\label{sec:Introduction}


The strong interaction among quarks and gluons is described by
Quantum Chromodynamics (QCD) that has two regimes; asymptotic freedom at short
distance and colour confinement at long distances.
At short distance, the interaction can be well described using perturbative methods. 
However, confinement, which is a non-perturbative phenomenon, is not 
very well understood. The study of quarkonia ($Q\bar{Q}$) involves
both the perturbative and non-perturbative aspects of QCD. 
The quarkonia are composed of heavy constituents (charm and bottom quarks) and their
velocity can be considered small allowing to use 
non-relativistic formalism~\cite{Povh:1995mua,Ikhdair:2005jf} in the study. 
In a simple picture,  a quarkonium can be understood as a heavy quark pair ($Q\bar{Q}$)
bound in a colour singlet state by some effective potential interaction.
In this bound state, the constituents are separated by distances much smaller
than $1/\Lambda_{\rm QCD}$ where $\Lambda_{\rm QCD}$ is the QCD scale.
%This interaction gets screened 
%in the presence of a deconfined medium like Quark Gluon Plasma (QGP), causing 
%the bound state to melt away and thus the quarkonia yields are suppressed in the 
%heavy ion collisions. This makes quarkonia an important probe of QGP. However cold nuclear matter 
%effects such as modification of parton distribution functions of nucleons inside nucleus also 
%affect their yields.


It is expected that the dynamics of strongly interacting matter changes
at temperatures and/or densities which are similar to or larger than
the typical hadronic scale.
It can be argued that under such extreme conditions, 
one should have the onset of deconfinement of quarks and gluons  and thus the 
strongly-interacting matter could then be described in terms of these
elementary degrees of freedom.
This new form of matter is known as 
quark-gluon plasma~\cite{Shuryak:1980tp,Satz:2011wf}, or QGP.
The support for the existence of such a transition has indeed been demonstrated 
from first principles using QCD simulation on lattice.
The heavy ion collisions provide experimental means to study the deconfinement transition 
and the properties of hot and dense strongly-interacting matter ~\cite{Satz:2000bn}
Significant parts of different experimental heavy-ion
programmes are dedicated to studying quarkonium yields. Such studies are 
motivated by the suggestion of  Matsui and Satz
that quarkonium suppression could be a signature of 
deconfinement~\cite{Matsui:1986dk}.
In fact, the observation of anomalous suppression of J/$\psi$ at SPS energies
was considered to be a key signature of deconfinement~\cite{Kluberg:2005yh}.


The $\Upsilon^{'}$s having three states with different binding
energies are far richer probes of the QCD dynamics in p+p and Pb+Pb collisions than
the charmonia states.
It is therefore important to achieve a good understanding of their
production mechanism in the vacuum as well as of how the nuclear effects in proton-nucleus
collisions affect them.
  At Large Hadron Collider (LHC) energy, the cross section of bottomonia production
is large and also the detector technologies enabled the study of various bottomonia 
states separately both in p+p and heavy ion collisions.
As proposed by different theories, bottomonium is an important and clean probe 
of hadronic collisions for at least two reasons. 
First, the effective field theory approach, which provides a link to first 
principles QCD, is more suitable for bottomonium due to better separation of 
scales and higher binding energies. Second, the statistical recombination effects 
are less important due to the higher mass of bottom quarks.  
Experimentally, the bottomonia are detected via their decay in dimuon channel which
which can be reconstructed with better mass resolution and smaller
combinatorial background due to higher mass as compared to other resonances. 
 All these properties make bottomonium a good probe of 
QGP formation in heavy ion collisions.

The detailed experimental study of the bottomonia states in p$\overline{\rm p}$ collisions 
were carried at Fermilab at $\surd s$ = 1.8 and 1.96 TeV ~\cite{CDF:1995gwi,CDF:2001fdy,D0:2005klj}.
LHC carried out the bottomonia study in p+p collisions at
$\surd s=$ 7 TeV~\cite{CMS:2010wld,CMS:2015xqv,ATLAS:2012lmu} and
13 TeV~\cite{CMS:2017dju,LHCb:2018yzj}.
There have been immense
experimental~\cite{Sirunyan:2017isk,Sirunyan:2017lzi,CMS:2018zza,Acharya:2019iur,ALICE:2018wzm}
and theoretical works~\cite{Strickland:2011mw,Song:2011nu,Kumar:2014kfa,Kumar:2019xdj} on
quarkonia modifications in Pb+Pb collisions.
The botomonia states in heavy ion collisions are suppressed with respect to their yields
in proton-proton collisions scaled by the number of binary nucleon-nucleon
(NN) collisions.
The amount of quarkonia suppression is expected to be sequentially ordered by the binding
energies of the quarkonia states.
 As the screening depends on the  binding energy the bottomonium states ($\Upsilon$(1S), $\Upsilon$(2S),
$\Upsilon$(3S), $\chi_{b}$, etc.) are extremely useful probes to understand the color screening
properties of the QGP.
The sequential suppression of the yields of $\Upsilon$(nS) states was first observed by
CMS at $\sNN =$2.76 TeV~\cite{Chatrchyan:2011pe,Chatrchyan:2012lxa}. Later, the results with
improved statistical precision have been reported by both the ALICE~\cite{ALICE:2018wzm}
and CMS Collaborations ~\cite{Sirunyan:2017lzi,CMS:2018zza} at $\sNN =$5.02 TeV.
The suppression of the $\Upsilon$(1S)
meson has also been studied at $\sNN =$200 GeV at Relativistic Heavy Ion Collider
(RHIC)~\cite{STAR:2013kwk}, although the 
bottomonia production cross section is small at lower energies.
There have been many reviews written on experimental and theoeretical developments on Quarkonia and
their modifications in Heavy ion collisions~\cite{Brambilla:2010cs,Andronic:2015wma,Rothkopf:2019ipj}.
The future measurements with high luminosity LHC can be found in Ref.~\cite{Chapon:2020heu}.

In this writeup, we review experimental and theoretical aspects of bottomonia production in p+p, p+A
and A+A collisions at RHIC and LHC energies.







