\section{Introduction}
\label{sec:Introduction}

Quantum Chromodynamics (QCD) describes the strong interaction among the
quarks and gluons via perturbative calculations utilising its property called asymptotic freedom.
On the other hand, these quarks and gluons are confined inside hadrons which are the colour singlet states.
Confinement is a purely non-perturbative phenomenon which is not very well understood yet. 
The study of quarkonia ($Q\bar{Q}$) serves as an effective 
tool to look at  both of these perturbative and non-perturbative aspects of QCD.
The quarkonia states differ from most other hadrons due to the small velocity, $v$ of the massive
constituents and thus can be treated using non-relativistic formalism~\cite{Povh:1995mua,Ikhdair:2005jf}. 
In a simple picture, one can think of a quarkonium as a heavy quark pair ($Q\bar{Q}$) bound
in a colour singlet state by some effective potential interaction, where the constituents are 
separated by distances much smaller than $1/\Lambda_{\rm QCD}$ where $\Lambda_{\rm QCD}$
is the QCD scale. This interaction gets screened 
in the presence of a deconfined medium like Quark Gluon Plasma (QGP), causing 
the bound state to melt away and thus the quarkonia yields are suppressed in the 
heavy ion collisions. This makes quarkonia an important probe of QGP. However cold nuclear matter 
effects such as modification of parton distribution functions of nucleons inside nucleus also 
affect their yields.
There have been immense experimental~\cite{Sirunyan:2017isk,Sirunyan:2018nsz,Acharya:2019iur,Acharya:2018mni}
and theoretical works~\cite{Strickland:2011mw,Song:2011nu,Kumar:2014kfa,Kumar:2019xdj} on
quarkonia modifications in PbPb collisions for which understanding of quarkonia
production in pp collisions is an important prerequisite.


It is expected that strongly-interacting matter shows qualitatively
new behavior at temperatures and/or densities which are
comparable to or larger than the typical hadronic scale.
It has been argued that under such extreme conditions
deconfinement of quarks and gluons should set in and the 
thermodynamics of strongly-interacting matter could then
be understood in terms of these elementary degrees of freedom.
This new form of matter is called
{\em quark-gluon plasma}~\cite{Shuryak:1980tp,Satz:2011wf}, or QGP.
The existence of such a transition has indeed been demonstrated 
from first principles using Monte Carlo simulations of lattice QCD.
The deconfinement transition and the properties of hot, strongly-interacting 
matter can be studied experimentally in heavy-ion collisions~\cite{Satz:2000bn}. 
A significant part of the extensive experimental heavy-ion
program is dedicated to measuring quarkonium yields since Matsui and Satz
suggested that quarkonium suppression could be a signature of 
deconfinement~\cite{Matsui:1986dk}.

In fact, the observation of anomalous suppression was considered to be
a key signature of deconfinement at SPS energies~\cite{Kluberg:2005yh}.

The primary motivation for studying high-energy heavy ion collisions is to better understand the hot and dense
matter produced in these interactions. Lattice quantum chromodynamics (LQCD) calculations indicate that at
sufficiently high temperatures a crossover occurs from hadronic matter to a strongly interacting system
of deconfined quarks and gluons known as ``quark-gluon plasma'' (QGP). One of the most prominent signatures of QGP
formation is that the production of quarkonia, the bound states of a heavy quark and its antiquark, is suppressed
with respect to expectations from scaling the yields in proton-proton collisions by the number of binary nucleon-nucleon
(NN) collisions. This suppression arises because the quarkonia binding is weakened by color screening caused by
the surrounding partons in the medium~\cite{Matsui:1986dk}. Therefore the extent of the quarkonia suppression is
expected to be sequentially ordered by the binding energies of the quarkonia states. Because of the binding energy
dependence of the screening, the bottomonium states ($\Upsilon$(1S), $\Upsilon$(2S),
$\Upsilon$(3S), $\chi_{b}$, etc.) are particularly useful probes to understand the space-time evolution
of the QGP. The sequential suppression of the yield of $\Upsilon$(nS) states was first observed by
CMS at $\sNN =$2.76 TeV~\cite{Chatrchyan:2011pe,Chatrchyan:2012lxa}. More recently, results with improved
statistical precision have been reported by both the ALICE and CMS Collaborations at $\sNN =$5.02
TeV~\cite{ALICE:2018wzm,Sirunyan:2018nsz,Sirunyan:2017lzi}. The suppression of the $\Upsilon$(1S)
meson has also been studied at $\sNN =$200 GeV at RHIC~\cite{STAR:2013kwk}.

The screening due to the QGP can also result in an azimuthal asymmetry in the observed yields of quarkonia. In non-central
heavy ion collisions, the produced QGP has a lenticular shape in the transverse plane. Consequently, the average path length
for quarkonia traveling through the medium depends on the direction taken with respect to this shape, with a larger suppression
in the direction of the longer axis~\cite{He:2015hfa}. The anisotropic distribution of particles can be characterized by the magnitudes
of the Fourier co-efficients (v$_{n}$) of the azimuthal correlation of particles~\cite{Voloshin:1994mz}. By studying the
azimuthal distribution of the quarkonia, it is possible to develop a more comprehensive understanding of the dynamics of their
production. The available experimental data, spanning from 0.20 to 5.02 TeV, have provided new insight into the thermal properties
of the QGP. In this section we review the current status of the experimental measurement of R$_{AA}$ and v$_{2}$ for
$\Upsilon$ states. The data from different experiments are comapred and physics insights from them is discussed.   






