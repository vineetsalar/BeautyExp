\section{Introduction}
\label{sec:Introduction}


The strong interaction amoung quarks and gluons is described by
Quantum Chromodynamics (QCD) that has two regimes; asymptotic freedom at short distance
and colour confinement at long distances.
At short distance, the interaction can be well described using perturbative methods,
but the confinement is a non-perturbative phenomenon which is not very
well understood yet. 
 The study of quarkonia ($Q\bar{Q}$) serves as an effective 
tool to look at  both of these perturbative and non-perturbative aspects of QCD.
The quarkonia states differ from most other hadrons due to the small velocity, $v$ of the massive
constituents and thus can be treated using non-relativistic formalism~\cite{Povh:1995mua,Ikhdair:2005jf}. 
In a simple picture, one can think of a quarkonium as a heavy quark pair ($Q\bar{Q}$) bound
in a colour singlet state by some effective potential interaction, where the constituents are 
separated by distances much smaller than $1/\Lambda_{\rm QCD}$ where $\Lambda_{\rm QCD}$
is the QCD scale.
%This interaction gets screened 
%in the presence of a deconfined medium like Quark Gluon Plasma (QGP), causing 
%the bound state to melt away and thus the quarkonia yields are suppressed in the 
%heavy ion collisions. This makes quarkonia an important probe of QGP. However cold nuclear matter 
%effects such as modification of parton distribution functions of nucleons inside nucleus also 
%affect their yields.


It is expected that the dynamics of strongly interacting matter changes
at temperatures and/or densities which are comparable to or larger than
the typical hadronic scale.
It has been argued that under such extreme conditions, 
deconfinement of quarks and gluons should set in and thus the 
dynamics of strongly-interacting matter could then
be understood in terms of these elementary degrees of freedom.
This new form of matter is called
quark-gluon plasma~\cite{Shuryak:1980tp,Satz:2011wf}, or QGP.
The existence of such a transition has indeed been demonstrated 
from first principles using simulations of lattice QCD.
 The deconfinement transition and the properties of hot, strongly-interacting 
matter can be studied experimentally in heavy-ion collisions~\cite{Satz:2000bn}. 
A significant part of the extensive experimental heavy-ion
program is dedicated to measuring quarkonium yields since Matsui and Satz
suggested that quarkonium suppression could be a signature of 
deconfinement~\cite{Matsui:1986dk}.
In fact, the observation of anomalous suppression was considered to be
a key signature of deconfinement at SPS energies~\cite{Kluberg:2005yh}.




The $\Upsilon^{'}$s having three states with different binding
energies are far richer probes of the QCD dynamics in p+p and Pb+Pb collisions than
the charmonia.
It is therefore important to achieve a good understanding of their
production mechanism in the vacuum as well as of how the nuclear effects in proton-nucleus
collisions affect them.
  At Large Hadron Collider (LHC) energy, the cross section of bottomonia production is large and
also the detector technologies enabled the study of various bottomonia 
states separately both in p+p and heavy ion collisions.
From a theoretical perspective, bottomonium is an important and clean probe 
of hadronic collisions for at least two reasons. 
First, the effective field theory approach, which provides a link to first 
principles QCD, is more suitable for bottomonium due to better separation of 
scales and higher binding energies. Second, the heavier bottom quark 
mass reduces the importance of statistical recombination effects.
Experimentally, there is a smaller background contribution in bottomonium
mass region and its decay in dimuon channel provides a reconstruction with better
mass resolution. 
 All these properties make bottomonium a good probe of 
QGP formation in heavy ion collisions.


There have been immense
experimental~\cite{Sirunyan:2017isk,Sirunyan:2017lzi,Sirunyan:2018nsz,Acharya:2019iur,Acharya:2018mni}
and theoretical works~\cite{Strickland:2011mw,Song:2011nu,Kumar:2014kfa,Kumar:2019xdj} on
quarkonia modifications in Pb+Pb collisions.
The botomonia states in heavy ion collisions are suppressed with respect to their yields
in proton-proton collisions scaled by the number of binary nucleon-nucleon
(NN) collisions.
The amount of quarkonia suppression is expected to be sequentially ordered by the binding
energies of the quarkonia states.
 Because of the binding energy
dependence of the screening, the bottomonium states ($\Upsilon$(1S), $\Upsilon$(2S),
$\Upsilon$(3S), $\chi_{b}$, etc.) are particularly useful probes to understand the color screening
properties of the QGP.
The sequential suppression of the yields of $\Upsilon$(nS) states was first observed by
CMS at $\sNN =$2.76 TeV~\cite{Chatrchyan:2011pe,Chatrchyan:2012lxa}. More recently, results with improved
statistical precision have been reported by both the ALICE~\cite{ALICE:2018wzm}
and CMS Collaborations ~\cite{Sirunyan:2017lzi,Sirunyan:2018nsz} at $\sNN =$5.02 TeV.
The suppression of the $\Upsilon$(1S)
meson has also been studied at $\sNN =$200 GeV at Relativistic Heavy Ion Collider (RHIC)~\cite{STAR:2013kwk}, although the
bottomonia production cross section is small at lower energies. 

In this writeup, we review experimental and theoretical aspects of bottomonia production in p+p, p+A
and A+A collisions at RHIC and LHC energies.







